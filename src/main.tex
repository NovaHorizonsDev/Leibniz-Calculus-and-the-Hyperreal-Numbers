%! Author = trevo
%! Date = 4/18/2024

% Preamble
\documentclass[12pt]{article}

% Packages
\usepackage{amsmath,geometry,setspace, enumerate}
\doublespacing

% Document
\begin{document}
\section*{\underline{Introduction}}
Our ideas revolving around Calculus, Philosophy, Law, and Theology are often so clouded that we can forget to remember the people who had these ideas and brought them to light We pass off these ideas as a fleeting thought that are taught within near hours, days, or as much as months.
Through this way of thinking, we forget to look at the foundations that took countless years and even lifetimes to construct out of what we believe to be nothingness.
What if I were to say that ideas included at the top were all formed or revolutionary changed by just one man over one lifetime?
This German mathematician, philosopher, and logician's name is Gottfried Wilhelm Leibniz.
This paper will outline and bring to light his contributions to the foundations and invention of Calculus, disagreements between him and Newton regarding Calculus, Leibniz's notation for Calculus, and some of his other work in other areas such as law, metaphysics, and much more.
Accurate recognition of one's work is critical in maintaining not only credibility over future pieces of work but also recognizing the accomplishments of one's work.
Understanding Leibniz's work allows us to also focus on the foundations of our modern societies and trace where many of our common ideas and innovations stem from.
Ultimately, by the end of this paper, one should have a better understanding of the impact Leibniz had on the modern interpretation of Calculus and the world entirely.


\section*{\underline{Infinitesimals}}

    \subsection*{Context}
Leibniz theorized that numbers existed that were infinitely small and/or infinitely great referred to as infinitesimal numbers.
Focused on this idea, Leibniz utilized this idea of infinitesimals to conceptualize the changes in small quantities that led to the idea of the derivative in Calculus (Goldblatt, 11).
Calculus to Leibniz was the examination and manipulation of infinitesimal values.
While his ideas about infinitesimals were often challenged during his time by many, including Issac Newton, ultimately with the mind of Abraham Robinson in the 1960s, his ideas were ultimately proven to not only exist but also be the fundamental ideas behind calculus, especially when it comes to the derivative and the integral.

\subsection*{Metaphysical and Theological Perspective}
Leibniz pondered the question on how we live in the most perfect of possible universes.
The idea follows that given some of what he referred to as 'substances' in a non-contradictory combination of existing substances comprise a possible universe in Christian Creationism.
For example, we will use $l, m, n, o, p$ as our substances.
Let's per se we add the following condition onto our substances:
    \begin{enumerate}
        \item Substance $l$ cannot coexist with substance $m$.
        \item Substance $o$ can only coexist with substances $l \ \& \ n$.
    \end{enumerate}
The goal of creation for God, according to Leibniz, is to create a universe with the most substance (most perfect) (Strickland, 29-31). \\
By this we could get a set containing all the possible.
Some elements can coexist together while some can't.
By creating such a collection of filters, the maximal element of such a filter would be the maximal filter that as such would yield the most perfect universe.
While with four substances it is straightforward, with an unknown amount(most likely infinite) on a grander scale, we can find such a filter used to derive the most perfect universe, which by Leibniz's reasoning on God's perfection in creation, is our universe.
This idea of a maximal element, maximum filter and filters would be where we start to set the scene for the roots of Calculus.

    \subsection*{The Hyperreal numbers and Continuity}
In 1966, these ideas about maximum filters and infinitesimals are what led mathematician Abraham Robinson to lay out the foundations of what we know today are the Hyperreal number system.
The Hyperreal numbers are an extension of the Real numbers symbolized as infinite sequences of Real numbers (Goldblatt,30-33).
This extension of the real number system allows us to understand continuity and differentiation in calculus the same way as Leibniz had developed roughly 300 years prior.
If two infinitesimals are infinitely close to each other (i.e. $ x \approx c$) then through a function($f$):  $f(x) \approx  f(c)$, (Goldbring, 81-82).
From here we can say that through the points the function is continuous.
We know that derivative exists for any continuous function in its domain.
Thus, this will bring us into Leibniz's work with derivatives.

\end{document}